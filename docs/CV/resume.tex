\documentclass[
	%a4paper, % Uncomment for A4 paper size (default is US letter)
	12pt, % Default font size, can use 10pt, 11pt or 12pt
]{resume} % Use the resume class

\usepackage{ebgaramond} % Use the EB Garamond font
\usepackage{hyperref}
\usepackage{enumitem}
\usepackage{amsmath}
\setlist[itemize]{noitemsep,topsep=0pt}
%\setlist{nosep}

%------------------------------------------------

\name{Anh Tran}

\address{SEO 733, 851 S Morgan St \\ Chicago, IL 60607}

\address{(872) 295 - 9030 \\ atran58@uic.edu} % Contact information

%----------------------------------------------------------------------------------------

\begin{document}

%----------------------------------------------------------------------------------------
%	EDUCATION SECTION
%----------------------------------------------------------------------------------------

\begin{rSection}{Education}
	
    \textbf{Rice University, Houston, TX} \hfill 2021\\
    B.S. in Mathematics\\
	\textbf{University of Illinois at Chicago, Chicago, IL} \hfill 2022 - (expected) 2027\\
    Ph.D. in Mathematics
\end{rSection}

%----------------------------------------------------------------------------------------
%	WORK EXPERIENCE SECTION
%----------------------------------------------------------------------------------------

\begin{rSection}{Experience}
    \textbf{Teaching Experience}\\
    $\circ$ Rice University: TA for MATH 354 Honors Linear Algebra, 356 Group Theory, 376 Algebraic Geometry, STAT 312 Statistics and COMP 382 Algorithms. Responsibilities included grading, holding office hours and assisting in lab sessions.\\
    $\circ$ UIC: TA for MATH 125 Elementary Linear Algebra, 210 Calculus III. Responsibilities include leading weekly sessions, grading, and office hours.\\
    \textbf{Organizational Experience}\\
    $\circ$ Organizer, Kobayashi-Hitchin correspondence reading group \hfill Spring 2025\\
    $\circ$ Organizer, Moduli of K3 surfaces reading group \hfill Fall 2024\\
    $\circ$ Co-organizer, Graduate Algebraic Geometry seminar \hfill 2024-2025\\
    $\circ$ Co-organizer, Graduate Number Theory seminar \hfill Spring 2024\\
    \textbf{Mathematics Research}\\
    $\circ$ Summer 2017 REU, University of Minnesota, Twin Cities. Worked on a question about the characteristic polynomials of certain pattern avoiding permutations. Report \href{http://www-users.math.umn.edu/~reiner/REU/GaetzHardtSridharTran_prob8_2017.pdf}{here}. 
\end{rSection}

\begin{rSection}{Conferences, Workshops}
    Unlikely Intersection -- Arizona Winter School \hfill March 2023\\
    Hodge theory and o-minimality -- CIRM, Luminy \hfill January 2024\\
    Summer Research Institute in Algebraic Geometry - Fort Collins, CO \hfill July 2025
\end{rSection}

\begin{rSection}{Graduate Seminar Talks}
    \textbf{Graduate AG Seminar.} Introduction to Hodge Theory -- Construction of moduli of K3 surfaces -- General curve of genus 6 -- Decompositions in derived categories -- Generic Torelli theorem -- Connectedness of Brill-Noether loci.\\
    \textbf{Graduate NT Seminar.} Root systems of algebraic groups -- Jacobian varieties. \\
    \textbf{Hodge Theory Seminar.} Mumford-Tate group -- Deligne's finiteness theorem -- Structure of period map -- Limit Mixed Hodge Structure -- Kummer surfaces -- Basics of D-modules -- Intersection cohomology -- Hodge modules over a curve. 
\end{rSection}
\begin{rSection}{Honors, Awards}
    2018 Hubert E. Bray Prize, given annually to the most outstanding mathematics junior.
\end{rSection}
\begin{rSection}{Technical Skills}
    Fluent in C, C++, Python. 
\end{rSection}

\end{document}